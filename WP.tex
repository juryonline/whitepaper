\documentclass[12pt]{article}
%\usepackage[a4paper, total={6in, 8in}]{geometry}
%\usepackage[a4paper, total={6in, 8in}]{geometry}
\usepackage[utf8]{inputenc}
\usepackage[english]{babel}
\usepackage{hyperref}
\usepackage{graphicx}
\usepackage{paralist}
\usepackage{enumitem}
\usepackage{parskip}
\usepackage{indentfirst}
\usepackage{amsmath}
\usepackage{amsfonts}
\usepackage{breqn}
\usepackage{amssymb}
\usepackage{amsthm}
\usepackage{yfonts}
%\usepackage{txfonts}
\selectlanguage{english}


\title{Jury.online}
\author{Alexander Shevtsov}
\date{\today\\v2.4}
\begin{document}
\maketitle
\tableofcontents 
\section{Introduction}
\subsection{History}
Deals and contracts are an integral part of human life. Today they come in many forms as a result of human and computer interactions.
However, since commodity trading markets evolved millennia ago in Mesopotamia until today trust remains paramount in business relationships
and global trade.

Yet, when business relationships are challenged or fail, parties may seek justice and  protection of their interests in courts arbitration
and private settlements. The development of global judicial systems and legislation has not kept pace with the progress of technological
innovation, especially in newly emerging markets and technologies. 

National laws are not yet corresponding to realities of the modern world, and existing legal cases have clogged the already overburdened
judicial process, which has coincided with a steep rise in the cost of legal services and litigation. 

Let’s have a glance at the contemporary problems of dispute settlement.

\subsection{Problems} Modern litigation or other kinds of dispute resolution are completely outdated and have a number of fundamental
disadvantages:

\begin{itemize}
		\item High price.  Professional legal services are extremely expensive, with preliminary consultations alone costing
				hundreds of dollars, while attorney’s fees for managing civil lawsuits account for thousands of dollars.

		\item Duration of proceedings.  As a rule, it takes several court sessions with 1-2 months intervals to solve a case. That, of course, is
				too long for most disputes.

		\item Execution of judgement.  Even if a judgement is delivered, its execution takes time and is carried out by third parties. The losing
				party may abscond, declare bankruptcy or otherwise escape from fulfilling its obligations.

		\item High entry requirements.  Most often, only big cases are considered in courts, and few people are ready to start serious litigation
				for disputing over some everyday matters, like poor-quality product or service. The cost and complexity of proceedings do not depend too
				much on subject of a deal.

		\item Jurisdictions.  State courts administer justice under the laws of a certain state which vary significantly in different countries.

		\item Political engagement and bias.  Courts are not always independent — often they are influenced by other institutions and people.

		\item Complexity.  Only few people can protect their own interests, for the rest trials are very complicated.

		\item Lack of choice.  There is no way to choose specific rules to be used for dispute resolution. Usually it is the national legislation of
				a certain country which cannot be altered by the parties. 
\end{itemize}

\subsection{Objectives} Due to the problems mentioned in the subsection above, we consider current execution and regulation of deals to be
completely outdated and want to change them by creating an interaction protocol for judges and parties of a deal, as well as a transparent,
secure, and convenient platform for making deals using blockchain and modern cryptographic systems.

\subsection{Blockchain} Blockchain technology became widespread after 2008\cite{nakamoto}. Blockchain is ledger of information blocks with built-in fault
tolerance and absence of centralized control. It can be thought of as a decentralized database with distinct rules of update. For today,
blockchain is the best tool for solving problems of trust and security.

The information stored in blockchain network is open, since common database simultaneously exists on all computers that are part of the
network. All changes in the blockchain are carried out stage by stage by means of transactions, each of them being open to everybody.
Blockchain cannot be corrupted, since all changes to be made should satisfy the rules, and the rules are predefined by the protocol
specification which is developed by community. Moreover, changing even a single unit of information requires changing the information on
every computer connected to the blockchain, which is impossible. 

Blockchain also involves transfer of value --- cryptocurrency. The transfer of cryptocurrency has all the advantages of the blockchain:
openness and unforgeability, while the transfer fee is much lower than that in banks.

Furthermore, blockchain gave rise to the technology of smart contracts, i.e. autonomous network agents capable of interacting with other
network members, including people. Smart contract operation is defined by the program code stored in the blockchain and open to everyone,
making it possible to realize complex interactions between network nodes. The course of such interactions is open and unchangeable.

\section{Description}
\subsection{Overview} Jury.online enables users to make deals which, in case of dissatisfaction of any party, are
considered by a panel of jurors delivering judgement in favor of one of the parties. Jury.online also gives any person with expertise in
certain field an opportunity to use their experience and knowledge for paid dispute resolution.

\subsection{Operation} A person willing to initiate a deal should open the website or mobile application and create a deal using one of the
ready templates and specifying all important details.

The deal is then placed into the blockchain as a smart contract. After tje deail is accepted by the other party, it
cannot be deleted or changed by anyone, even the platform administration has no control over this smart contract. In
addition, the deal is assigned with a link which leads to the website page displaying information about the deal. The
initiator sends this link to the counterparty or makes it accessible to the public if they do not yet know who is going
to be the counterparty. For example, a deal is intended for product or service delivery, but the initiator does not know
who is going to perform it.

When creating a deal, the initiator indicates a sum they are ready to pay for the job (or want to get, if the initiator
is a contractor).  This sum, converted into cryptocurrency, is sent to the balance of the smart contract, i.e.
deposited. This money cannot be withdrawn until the deal is completed, regardless of whether it was successful, whether
the parties were content about its execution or need to start legal proceedings. Part of the balance is allocated to
resolve possible disputes, i.e. to pay court fees (it’s not the only option, fees can be paid separately).

After the counterparty confirms its consent to the terms and conditions of the deal and specifies the information necessary, the deal is
considered to be concluded.

After that, the parties have certain period of time to fulfill their obligations. In the event each party is satisfied with the result, the
money from smart contract of the deal is transferred to the counterparty (or elsewhere, if specified by the deal), and the deal is
considered to be successfully completed.

In the event one of the parties is not satisfied with the deal execution and believes that the counterparty has not fulfilled its
obligations, the deal is sent for judges’ consideration. Any party may initiate a dispute and send the deal to judges, but as a rule it is
the party that deposited money.

When initiating a dispute, the parties have certain period of time to set forth their arguments and comments on the issue. After that, the
deal is sent to judges for consideration.

Form of dispute resolution may vary, but usually the system chooses $n$ random judges who receive anonymized information about the deal and
take the decision by an absolute majority of votes. 

Judges are provided by a source called a judge pool: in the simplest it’s done by Jury.online, but third parties specializing in a certain
sphere may also offer dispute resolution services.

The identity of judges is unknown to the parties, but their competence is. Judges have a fixed period of time to take a decision.
Information about the decision of a particular judge is encrypted and unavailable to other judges.

There are other forms of judging accessible: instead of several random unknown judges the parties may agree on a specific judge they
consider fair and uncorrupted. 


Most deals imply performance of contractual obligations for a payment. In this case, one party has automatically fulfilled its obligations
as it has transferred the payment to a smart contract, and it is the only party interested in dispute. However, we do not want to limit the
range of possible deals, for example, service for service or other options, so it is possible to choose who and when is to pay for possible
litigation.

\subsection{Use cases}
Here are several examples of using jury.online platform and protocol:
\begin{itemize}
		\item The party is a company
				that needs to design a logo or write a text by certain terms of reference. The company is quite well-known and pays well, thus expecting
				proper quality of the works performed. The counterparty is an artist without any outstanding works who hasn’t used jury.online service
				before. The party wants to hire the counterparty but needs guarantees that the job will be done on time.

				In this case, the party is the one interested in the potential dispute, as it may not like the result. Chance of dispute initiated by the
				counterparty is small, since the money is kept on a smart contract and the party has no control over it.

				The party creates a deal at jury.online, with money for the work deposited to the deal balance. When creating the deal, the party specifies
				that in case of a dispute, the counterparty covers 80\% of litigation expenses, making a payment in advance. This ensures that the party will
				bear no losses if the counterparty does not do the job. The party itself pays 20\%, being a guarantee for the counterparty that the party
				will not dispute the deal regardless of its result. This ratio (80\% to 20\%) is not mandatory and may vary.

				Moreover, the counterparty can make an additional deposit as a guarantee of immediate high-quality result. In this case, the party may pay
				for the proceedings.

		\item Counterparties are residents of two faraway countries representing big companies and willing to execute a deal, e.g. goods delivery.
				However, they understand that if something goes wrong, fees for litigation in other country will be significant. In this case, the deal with
				classic dispute resolution is too risky, and the game is not worth the candle.


		\item Payment prolongation and warranty cases.  Imagine you want to buy a TV, you find a seller, buy the TV and bring it home. It’s working
				fine, but in a week it fails to show anything, and the seller is already gone or directs you to the repair shop. Use of Jury.online platform
				would force the seller to be interested in resolution of such a case.

				Money can be transferred after the product has been delivered and used for some time, or any other work can be paid for after proper testing
				and examination. The payment can be divided into parts with the next part coming only after the previous obligation is fulfilled.

		\item Jobs for experts.  There are many highly-qualified diverse specialists all around the world who have skills, experience and personal
				point of view concerning field-specific questions which are out of the competence of regular jurors. Thousands of these experts are
				currently unemployed and have no chance to take part in dispute regulation and get money for their job though they are knowledgeable enough
				in the topics considered in courts. Jury.online helps people earn money with their knowledge, competence, skills, experience and
				independence of mind. 
\end{itemize}
The list of applications may be continued with million of other cases.
\section{Jury.online protocol}
The protocol is a description of a smart contract operation, its specification for deal execution and resolution. The protocol defines
interaction of counterparties, judges and pools of judges, as well as requirements for information encryption and side channel
communication. 

The protocol is open to the public which can adjust and develop it, search for possible vulnerabilities and make suggestions. 

Described below are general ideas and requirements, while full description is available in the technical whitepaper.

Clear understanding of the following subsections requires familiarity with cryptographic primitives, such as public-key
cryptography, cryptographic hash functions, and pseudorandom number generators. Explanation of these topics goes beyond
the purpose of this document and Wikipedia articles provide enough insight for further reading  \cite{publickey, hash, prng}.


\subsection{Requirements for participants}
All participants — counterparties, judges, pool operators --- must have a pair of keys used in asymmetric cryptosystem for their
identification and to sign their actions. Such a key pair is given to every owner of an ``account'' of a typical blockchain.
\subsection{Attachements}
Counterparties of the deal need a secure way to share files and documents, and these attachments later must be exposed to the judges.
Moreover, the jurors need complete information about all actions, and they also need to be sure that the files presented were really sent by
a certain counterparty. Fortunately, modern cryptography successfully solves this problem via good old asymmetric cryptosystems. During a
dispute resolution, the counterparties decide which documents are vital for the correct outcome and disclose them to the jurors.
\subsection{Smart contract of a deal}
Described below are requirements for a smart contract, information it stores, functions to be called and participants to call them when the
deal becomes active.

Smart contract stores the following information:
\begin{enumerate}
		\item Counterparty identificators
		\item Subject of the deal, links to related documents and attachments.
		\item Starting time of the deal, time for execution, time for acceptance, the moment of dispute fee payment.
		\item Counterparties’ deposits and collateral for dispute resolution.
		\item Type of dispute resolution, e.g. identificator of the pool of judges for random judges.
		\item Identificators of other smart contracts used in the protocol:
				\begin{enumerate}
						\item rating smart contract (Rater)
						\item chooser responsible for choosing a judge, based on a random number generator (RNG)
				\end{enumerate}
\end{enumerate}

This data is vital for smart contract operation, but typically it would hold more information, e.g. information about public visibility of
the deal at the website or number of possible appeals.


\subsection{Multiparty deals}
In fact a deal is just an obligation statement agreed with all involved parties. Generalization of the case with two
parties to the greater number of participating sides encounters no new fundamental problems.  However, on the contrary
to the case with only two parties, multilateral deals must prescribe more detailed behaviour in possible outcomes,
because determining the ``wrong'' in general would not result in automatical determination of the ``right''.

For instance imagine a customer who needs a website. He finds a designer, backend developer and frontend
developer, and all of them agree to work together. Customer makes a deposit for complete website partitioned for all
workers. In case one of the developers disappears and the rest workers fullfill their obligations, it's not clear how
to handle the deposit. 

Nevertheless, with proper description of the outcomes, multiparty deal can be treated in the same way as a two-party deal. So if rules for all
the outcomes are set beforehand, like ``in case of one of the worker leaves, or doesn't finish work on time, then the
rest receive $80\%$ of their payments''.

If the parties don't form interconnected obligations, then relations between parties should be pairwise with a
separate deal for each pair.

Summing up, jury.online protocol allows creation of multilateral deals, but its use is more complicated and we hardly see
it as a main purpose of the system.

\subsection{Requirements for pools}
Pools should maintain up-to-date lists of active judges who are online and ready to consider disputes. In order for a smart contract to know
them, part or all of the information about the list should be kept in the blockchain. 

Depending on the price of transactions in the blockchain, there are at least two mechanisms of storing up-to-date information:

\begin{itemize}
		\item Storing identificators of all judges directly in the blockchain. In this case, however, keeping the list updated might be expensive as
				it requires sending transactions to the blockchain. 
		\item Publishing cryptographic hash of the list in the blockchain. While the list is
				provided by the pool on their website, Jury.online application or a user themselves can verify that the provided list corresponds to the
				published hash.
\end{itemize}

\subsection{Selection of judges}
Basic services include random selection of judges for dispute resolution. Pools provide up-to-date lists of active judges. In case of a
dispute, a pseudorandom number generator selects the required number $k$ of specific judges from this list who receive information about the
deal. Random number generator operates via smart contract so that it is not controlled by any party. However, for operation it needs some
initial state that is defined by a numeric parameter called seed. Since all information in the blockchain is open and accessible to any
user, a pool operator or a party that knows the seed can adjust the judges’ order so that a certain deal is considered by certain judges who
may be in collusion with a party. Therefore, seed cannot be calculated on the basis of any public information but should use parameters
provided by the counterparties. Since counterparties to a dispute pursue opposite goals, they are interested in safe and high-quality seed
that would lead to an unpredictable judge choice. 

\subsection{Summary of the working scheme}
\begin{enumerate}
		\item Counterparties generate random integer numbers $a$, $b$ in the range of $0..2^{256}-1$.
		\item Counterparties calculate the cryptographic hash of these numbers:  
		\[A=\texttt{hash}(a)\]
		\[B=\texttt{hash}(b)\]
		\item Counterparties send $A$, $B$ to the smart contract of the random number generator.
		\item Counterparties send $a$, $b$ to the smart contract of the random number generator, and it checks that previously sent hashes are indeed of these numbers:
				\[A =\texttt{hash}(a)\]
				\[B =\texttt{hash}(b)\]
				If one of the equalities is false, it means that a counterparty
				is trying to deceive the system and choose non-random judges.
				In this case, the counterparty becomes a losing party, and the
				dispute is resolved in favor of the other party. 

		\item Smart contract calculates seed using both numbers $a$, $b$, for instance:
				\[\texttt{seed} = \texttt{hash}(a\Vert b),\] where $\Vert$ denotes concatenation.
		\item Based on the calculated seed, the generator produces the required random numbers $i_1, i_2, \dots , i_k$ identifying judges from the pool.
				\[[i_1, i_2, \dots , i_k ] =\texttt{prng}(\texttt{seed}, k)\]
		\item After calculating identificators of random judges, smart contract of the random number generator sends them to the smart contract of the deal.
\end{enumerate}
What is ensured this way? As order of sending numbers $a$, $b$ cannot be regulated, the counterparty that is second to
send them can adjust its numbers so that the seed has a certain special form resulting in generation of identificators
of the judges affiliated with it. Thus, hash sending guarantees that after the numbers are selected they cannot be
changed neither by the deal party nor by the pool operator.

\subsection{Judges’ verdicts}
Economic and rating motivation forces judges to investigate and resolve disputes fairly and correctly, rather than to randomly pass their
verdicts. Judges should not know verdicts of other judges to prevent them from taking the decision voted for by a majority. Therefore, a
mechanism for hiding the verdicts must be used, and verdicts cannot be stored in the blockchain as the number of for/against votes.

Jury.online uses probabilistic encryption%\cite{}<++>
This method has one important feature: cipher text doesn't

In short

The algorithm used her
Verdict is encrypted with some additional randomly generated data — ``salt''. This data is generated by the parties of the deal. Parties to
encrypt the verdict are chosen in rotation.

Party uses symmetric-key algorithm to encrypt the salt and to store it in the smart contract of the deal. Later the party will disclose the
encryption key and reveal the verdict. Refusal to disclose the key would mean that this party lost the dispute.

Then a party encrypts the salt using judge’s public key and sends it to the judge via side channel, so it’s not published in the blockchain.
Judge can reveal the salt and make their verdict open, however it can be done only by publishing their private key, so the judge will lose
control of their account and all funds at its balance.

Another approach is to use probabilistic encryption, though it’s rather complicated and is therefore described in the technical whitepaper
and protocol specification there. A significant advantage of this method is that there is no need in side channel communication.

\subsection{Ratings}
Rating system is vital for correct operation, because dispute resolution price is based on the judges’ competence. Estimation of someone's
competence is a very nontrivial task, and it’s evaluated by a number of statistical methods that provide resulting numerical parameter —
rating. 

As we want rating to be fair and uncontrollable by anyone, it’s implemented via smart contracts, so it’s automatically recalculated after
any of the parameter has changed. Along with a rating system, there are some open metrics which can be used for judge assessment and choice.
E. g. a party wants a potential dispute to be settled by a group of jurors who have average response time less than 2 hours and total money
earned more than equivalent of $\$10\ 000$.

List of these metrics is given below. 
Counterparty participant:
\begin{itemize}
		\item Number of deals they participated in.
		\item Number of deals resulting in a dispute.
		\item Number (ratio) of disputes won.
		\item Money spent on deals/disputes.
		\item Frequency of deal creation
\end{itemize}
Judge participant:
\begin{enumerate}
		\item Number of disputes they participated in.
		\item Number (ratio) of disputes won, i.e. number of disputes where they took a decision that became final.
		\item Amount of money earned through the system.
		\item Average (median, modal) time of decision making.
		\item Average (median, modal) availability, i.e. how often they are online and ready for dispute resolution.
		\item Pairwise judge-counterparty metrics, i.e. how many times this judge participated in disputes of a certain counterparty.
		\item Number of appeals won or lost.
\end{enumerate}

Both counterparties and judges rate each other, but these metrics are not used to evaluate anyone’s competence and are shown just as additional parameter.

\section{Jury.online platform}
Jury.online implements the above-mentioned protocol on the basis of Ethereum blockchain connecting counterparties and judges. 
\subsection{Operation}
Unfortunately, working with blockchain is still challenging for people unaware of technical details. For normal operation, one needs to have
the entire Ethereum blockchain which currently occupies $290$ GB of memory\cite{etherscan}. Of course, it is too much for an average user, not to mention
mobile devices. Therefore, jury.online operates as an intermediary enabling users to send transactions to the blockchain. Transactions
cannot be changed by jury.online due to the absence of users’ private keys.

Jury.online keeps its own nodes and servers for flawless service operation, and the user doesn’t have to manually call contracts functions
through parity or mist. The user visits the website or mobile application, performs an action which is translated into blockchain
transaction and signs it by their private key stored on their side.

\subsection{Connection of pools}
Jury.online provides various organizations, companies and groups of specialists with the opportunity to offer their dispute resolution
services. After verification and audit, jury.online ``issues a license'' to the pool and includes this pool into the list of those publicly available
when a deal is created. Jury.online publishes statistics and average user ratings when using this pool. In case of unsatisfactory operation,
the license is revoked. Also note that pool of judges is selected by counterparties and not imposed by jury.online.

\subsection{Economy}
Cryptocurrencies transfer arises in the workflow in following forms:
\begin{enumerate}
	\item Transaction fee of the blockchain\footnote{If there is any fee for transaction in the underlying blockchain}. 

		This part is paid by the initiator of the deal. By agreement with the other counterparty this
		fee can count as a part of the deal sum. 
	\item Deal amount. 

		Both counterparties may deposit funds to the smart contract of the deal. Hence, counterparty that
		is meant to receive the deal amount may provide a pledge to show their serious altitude to the deal.
	\item Dispute resolution payment

		This part must be paid in order to start the dispute. As in dispute there is at least one dissatisfied party, we
		expect exactly this party to provide this payment.
\end{enumerate}
The first two items are nominated in the internal blockchain cryptocurrency. Jury.online has no control on these amounts and takes no commission from these parts.

Jury online takes commission only from dispute resolution payment, the third item from the list.
Dispute resolution fee is paid in Jury.online Tokens (JOT) which are issued at the ICO. Fee is only charged in case of
resolution, so a deal without a dispute takes no other fee than that for blockchain transaction.

So jury.online receives the same fee both for the $\$10\ 000$ deal and $\$100$ deal. However, we expect that $\$10\ 000$ deal
requires more qualified jurors for dispute resolution, hence it will have larger dispute resolition payment.

Moment of fee payment is specified beforehand, so a deal can be created with tokens deposited for a potential dispute, or tokens can be
purchased when the dispute is started. If the deal is successfully executed without dispute, tokens are returned back to the party that
provided them. 

\subsection{Jury.online commission}

This commission is used for platform maintenance and further development. Maximum commission is 20\%, while minimum is 0\%. 

Rules are as follows:
\begin{itemize}
		\item First $1000$ resolutions have $0\%$ commission, with all the tokens deposited going to the judges
		\item Next $9000$ resolutions require $10\%$ commission
		\item Subsequent resolutions require $20\%$ commission, though this value may be decreased depending on the circumstances
				(e.g. discount, sale, etc.).
\end{itemize}

JOT is also used to pay for pool audit and verification, as well as for the issue of a ``pool license''.

JOT can be bought at cryptocurrency exchanges or directly from the Jury.online (if it has any JOT supply).

\subsection{Appeals}
Anything in our world can be thought from a different point of view, no judge can satisfy everyone's expectation
and fit everyone's concept of justice. Judges' work is to provide unbiased and just verdict, but the judge is not a
superhuman, may make mistakes as any other person. However, review of a verdict by ``more qualified'' juror and repeating
this process of reviewing judge's verdict by a superordinate judge makes the probability of error sufficiently small. 

Counterparties of the deal may agree on the potential appeals for the deal. Appeal is in essence just a dispute about the
verdicts of the initial dispute. Every judge providing his verdict supplies it with his arguments and comments, why and
how he came up with his decision. Later, a judge of a higher rank involved in appeal reviews these comments and
arguments and makes his own decision.

If a judge's verdict is found to be wrong by supreme instance, judge loses his payment and rating. Such a loss is more
severe than that of just being wrong.

%\subsection{Jurisdictions}



\subsection{Protocol development, research}
Presented protocol is not final, it will be improved. Current version is blockchain-agnostic, and our implementation uses Ethereum, as it’s
the most convenient blockchain with Turing-complete smart-contacts today. While the blockchain technology is widely spreading today, we may
use some other blockchain if its properties fit the protocol needs better than Ethereum.

\section{Features}
\subsection{Escrow}
Escrow is the most important feature. Money is not transferred immediately but stored by a third party, in this case, on a smart contract in
the blockchain that is completely independent and operates based exclusively on the code of that contract.

\subsection{Jury market}
We turn court judgements into a market. You can always figure out what is more expensive: to start litigation or not to start. Jurors
provide price for which they are ready to work, so the counterparties know in advance how much a potential dispute would cost and may
accurately estimate their risks.

\subsection{Jury pools}
Use of the platform and protocol is impossible without judges, so we give third-party companies the opportunity to provide judges for dispute resolution (even quite specific ones).

\subsection{Customization of dispute resolution}
As it was stated in the problems list, modern disputes are hardly connected with jurisdictions, and people have no choice in how their
dispute should be resolved. jury.online offers a range of opportunities:
\begin{enumerate}
		\item Judges from a general-purpose pool working on disputes that require no particular skills or specific knowledge. These judges remain anonymous to prevent colluding.
		\item Judges who are specialists in a certain industry. Of course, their services would cost more compared to those of non-experts.
		\item A person or a group of people known to be impartial professionals in a certain sphere credible in opinion of all the counterparties —
				these judges are open and their names are not kept in secret.
		\item Each counterparty may have their own judges, more or less affiliated
				with them. However, the other party is entitled to reject some of the persons on the list.
\end{enumerate}
Additional features:
\begin{itemize}
		\item Opportunity to reject a number of judges proposed, if the counterparty considers they are not fair. The greater the number of judges rejected, the higher the cost.
		\item Before the deal is started, counterparties may provide their set of laws, i.e. set of characteristics that should be used to estimate the deal execution.
		\item Pairwise judge-counterparty statistics to show that a certain judge is not affiliated with a certain counterparty.
\end{itemize}

\subsection{Integration}
There are plenty of companies that need dispute resolution services: online stores, freelance platforms, marketplaces. For instance, instead
of having their own dispute resolution department Amazon or Upwork could outsource this service to jury.online platform.

Any party with such a need can assure their customers that the service provided is fair, and should any dispute arise, it would be
considered by unprejudiced jurors. 

Set or rules for connecting the service to jury.online system is being developed now and will later be published at jury.online website.

After the integration with jury.online third parties may list dispute statistics information of a certain user in order to have additional
metrics for reputation systems and proficiency evaluation. 

\subsection{Security}
The system operates based on smart-contract code, not human action which can be easily affected. It is known to any party and cannot be
changed. This also applies to money transfer: money is held on the smart-contract and cannot be withdrawn in any way except after successful
deal execution or dispute resolution by jurors from a pool agreed beforehand.

Every human action involves use of cryptographic keys of a participant that unambiguously identify the user. All actions are stored in
blockchain transactions, and blockchain network has already proven its unwavering reliability when used for any purposes.

\subsection{Paranoid mode}
Paranoid mode is for the users that don’t want to entrust our service with anything. For example, they may be afraid that we could censor
their transaction and their action would not be broadcasted into blockchain on time, which is the only possible way to interrupt the
contract operation, as jury.online cannot fake a blockchain transaction and corrupt the user’s intent.

Paranoid mode involves a toolkit with open source code along with a set of instructions on how to use the platform without any third party.
However, this way requires user to have a blockchain node and to call all the smart contract functions by themselves, which is rather
nontrivial for an inexperienced user.
\subsection{Statistics}
Jury.online platform operation involves a statistics contract which counts the following data about overall platform operation:
\begin{itemize}
		\item Number of deals created
		\item Number of disputes opened
		\item Total sum of collaterals transferred 
		\item Total sum of tokens used for dispute resolution (total sum of JOTs earned by jurors).
\end{itemize}

\subsection{Audit}
We provide selective audit of deals, judges and disputes by administration of jury.online. We would consider random deals and disputes and
rate their resolutions by ourselves. It would not change ratings, as rating system is based on decentralized smart-contracts, but any users
at the website or application would see additional verification mark of our own assessment at a person’s profile. 

\section{Initial coin offering}
ICO will have the following stages:

\begin{enumerate}
		\item Presale. 
				Up to the $20\%$ of all tokens available to public will be allocated with a $30\%$ discount during an $11$-day presale.

				Presale limit: $20\%$ of all tokens for sale $=4\ 200\ 000$ JOT (up to $14\ 000$ ETH)

				Price: $1$ ETH = $300$ JOT ($1$ JOT $\approx 0.0033$ ETH).

				Minimal amount of tokens to be bought: $10\ 000$ JOT ($\approx 33$ ETH).

				Dates: 23 October 14:00 UTC -  03 November 2017 14:00 UTC.

		\item Sale
				Price: $1$ ETH = $210$ JOT ($1$ JOT $\approx 0.0048$ ETH).

				Minimal contribution: $5$ JOT. 

				Time bonuses for early contributors will start from $20\%$ discount on the first day with linear daily decrease by $\frac{2}{3}$\% ($2$ percent for $3$ days).

				Dates: 13 November 14:00 UTC - 13 December 14:00 UTC.
\end{enumerate}

Total maximum supply of JOT: $30\ 000\ 000$.

For every $7$ JOTs sold through ICO jury.online issues $3$ JOTs for itself. 

Thus, up to $21\ 000\ 000$ JOT will be available for public and up to $9\ 000\ 000$ JOT will be reserved by jury.online.

$21\ 000\ 000$ JOT available to public is divided into:
\begin{enumerate}
		\item Presale supply: $4\ 200\ 000$
		\item Sale supply: $16\ 800\ 000$
\end{enumerate}
ICO minimum cap is set for overall crowdsale, sum of sale and presale stages.  \[\text{ICO success: }3\ 000\ 000 \text{ JOT}\] 
Which is equal to $10\ 000$ ETH $\approx \$3$ mln during presale, $14\ 285$ ETH$\approx
\$4.29$ mln during sale; with $1$ ETH $\approx \$300$.

Tokens can be bought for ETH and BTC, as well as for fiat money.

Token transfer would be available only after the end of ICO.

The start and the end of each stage will be linked to specific Ethereum block numbers and will later be published at the website.

Tokens reserved by jury.online will be distributed as follows:
\begin{itemize}
		\item $20\%$ go to the jury.online team and will be blocked for a period of $6$ months. After that period, $10\%$ of the total amount blocked will be
				unblocked every month, with the whole sum therefore unblocked in $16$ months after the ICO end.

		\item $7\%$ go to advisers and for bounty programmes.
		\item $3\%$ remain as a stock of liquidity to ensure flawless and independent operation of jury.online before token enters cryptocurrency exchanges
				and the market stabilizes, as well as for compensations in exceptional cases.
\end{itemize}

\section{Roadmap and milestones}
\begin{itemize}
		\item Autumn 2017: ICO

				Development organization
		\item Spring 2018: Web version alpha release

				%\nopagebreak
				Basic features (deals with random judges, one pool) released.
				%\nopagebreak

				Testing, start of operation, creation of “general-purpose” pool.
		\item Summer 2018: Mobile applications release.
		\item Summer 2018: Implementing the full set of features (various resolution types, paranoid mode).
		\item Summer-Autumn 2018: Third-party pools opening.
		\item Autumn 2018: Development of third-party services integration infrastructure.
		\item Winter 2018-2019: Start of integration with third-party services (e.g. Upwork, Amazon).
\end{itemize}

%\section{In-depth analysis}
%This section is intended for the versed 
%\section{Pitfalls}
%\subsection{Sybil attack}
%\subsection{Sphere of application}
%We have no illusions: we don't expect our platform to suit every deal in the world.

%Despite a plenty of problems in modern judical system, it serves its roles in a range of cases. These cases include
%months longing court sessions with elaborate examinations of every argument; expertises by institutions;
%refereneces to the precedents occured before, or simply cases in which judges can't remain anonymous after they prove
%their competence. Sometimes classic court system is the only way to find out who is right and who is wrong.
%
%We don't expect multimillion deals to be executed using jury.online platform. Such deals invole It simply is not designed for such
%purposes. We understand, that a long, thorough case with 
%
%First of all we aim to change dispute resolution of \emph{simple} deals, deals in which common sense along with some
%knowledge in a certain field is enough to find the right and the wrong. 
%
%One can imagine case with an attacker registering a lot of fictious judges to get needed outcome for a certain dispute.
\section{Epilogue}
Smart contracts are digital contracts, written in code with clauses enforced automatically. World just starts to explore how to apply this technology. 
Yet, smart contracts have a lurking problem: transforming human intention into a smart contract code is a great challenge. And as our world
is getting more complicated so do our desires, such a complication would require more refined and powerful techniques.

Luckily, when properly motivated, humans can be intelligent, flexible, fair, and reasonable: exactly the qualities that bad code lacks. This
suggests that until we know how to express our desires in a not human-oriented way, we have to create systems which benefit from both human
and machine expertises to maximize the result.

So Jury.online is neither a new blockchain nor an old service which attempts to integrate blockchain. Jury.online is a project which aims to
use the full power of contemporary technologies to solve problems that were considered unsolvable for decades.

%\begin{thebibliography}{9}
%\end{thebibliography}
\bibliographystyle{plain}
\bibliography{references.bib} 
\end{document}
